%%%%%%%%%%%%%%%%%%%%%核心代码来自http://tex.stackexchange.com/a/159234
\documentclass{standalone}%[varwidth=18cm,] mutlti=true [tikz]
\RequirePackage{luatex85}
\def\pgfsysdriver{pgfsys-pdftex.def}
\usepackage{luatexja-fontspec}
\usepackage{tikz}
\usetikzlibrary{shapes,
                calc,
                fpu,
                patterns,
                decorations.pathreplacing,
                arrows,
                shapes.geometric,
                decorations.pathmorphing,
                spy,
                backgrounds,
                shapes.callouts,
                decorations.fractals,
                decorations.markings,
                angles,%http://tex.stackexchange.com/questions/219038/tikz-draw-angle-with-label-between-lines
                quotes,
                math
                }
\usepackage{pgfplots}
    \usepgfplotslibrary{fillbetween}
\usepackage{pdftexcmds}
\usepackage{xparse}
\usepackage{etoolbox}
\usepackage{animate}
\usepackage{amsmath}
\usepackage{ifthen}
\usepackage{tagging}
        %\usetag{AuxiliaryLine}%显示点之间的辅助连线
        %\usetag{DraftNode} %显示所有标记 变量
        %\usetag{DraftGrid} %显示网格辅助线
        %\usetag{DraftCirclePoint}%显示点的位置
\usepackage{fp}
\usepackage{graphicx}
        \DeclareGraphicsExtensions{.pdf,.png,.jpg,.mps}
%\usepackage{lua-visual-debug}

%%%%%%%%%%%%%%%%%%%%%%%%%%%%%%%%%%%%%%%%%%%%%%%%%%%%%%%%%%%%%%%%%%%%%%%%%%%%%%%
%%%%%%%%%%%%%%%%%%%%%%%%%%%%%%%%%%%%%%%%%%%%%%%%%%%%%%%%%%%%%%%%%%%%%%%%%%%%%%%
%%%%%%%%%%%%%%%%%%%%%%%%%%%%%%%%%%%%%%%%%%%%%%%%%%%%%%%%%%%%%%%%%%%%%%%%%%%%%%%
%%%%%%%%%%%%%%%%%%%%%%%%%%%%%%%%%%%%%%%%%%%%%%%%%%%%%%%%%%%%%%%%%%%%%%%%%%%%%%%
%变量定义




%%%%%%%%%%%%%%%%%%%%%%%%%%%%%%%%%%%%%%%%%%%%%%%%%%%%%%%%%%%%%%%%%%%%%%%%%%%%%%%%
%%%%%%%%%%%%%%%%%%%%%%%%%%%%%%%%%%%%%%%%%%%%%%%%%%%%%%%%%%%%%%%%%%%%%%%%%%%%%%%%
%%%%%%%%%%%%%%%%%%%%%%%%%%%%%%%%%%%%%%%%%%%%%%%%%%%%%%%%%%%%%%%%%%%%%%%%%%%%%%%%
%%%%%%%%%%%%%%%%%%%%%%%%%%%%%%%%%%%%%%%%%%%%%%%%%%%%%%%%%%%%%%%%%%%%%%%%%%%%%%%%
%%载入命令 设置
\makeatletter\input{DEConfig/Font.set}\makeatother %载入字体设置
%\makeatletter\input{DEConfig/TheBBCstyleCUG.set}\makeatother %BBC LOGO 形式 CUG LOGO
%\makeatletter\input{DEConfig/TikzVariables.tikzconfig}\makeatother %载入全局变量
\makeatletter\input{DEConfig/FillPattern.tikzconfig}\makeatother %填充效果
%\makeatletter\input{DEConfig/HelpLines.tikzset}\makeatother %坐标辅助线
%\makeatletter\input{DEConfig/LaserBeamStyle.tikzset}\makeatother %线两边模糊效果
%\makeatletter\input{DEConfig/CoteFunction.tikzcmd}\makeatother      %CAD尺寸标记效果
%\makeatletter\input{DEConfig/DrawNewtonRing.tikzcmd}\makeatother    %画完整牛顿环0-\MaxRings环,并标识出每环水平方向左右两边 及左上 右上的标记点
%\makeatletter\input{DEConfig/IndicateRingNumber.tikzcmd}\makeatother %显示怎么数环
%\makeatletter\input{DEConfig/DmDsLmLsCote.tikzcmd}\makeatother      %标记Dm Ds Lm Ls 和它们之间的关系
\makeatletter\input{DEConfig/CalculateFunction.tikzcmd}\makeatother      %计算两点间距离
%\makeatletter\input{DEConfig/DrawCross.tikzcmd}\makeatother         %画目镜十字线命令使用了线两边模糊效果,因此LaserBeamStyle.tikzset应先加载
%\makeatletter\input{DEConfig/ShowHowToTakeReading.tikzcmd}\makeatother %演示如何读数 m=\m,s=\s
\makeatletter\input{DEConfig/TikzPictureStyleDefine.tikzset}\makeatother %预定义的形式

\makeatletter\input{DEConfig/TikzTheBasicShape.tikzset}\makeatother %斜面和滑块

\def\AuxiliaryNodeFontSize{2} %单位为pt

\pgfmathsetmacro{\Inclination}{15}  %斜面倾角 转换为弧度
\pgfmathsetmacro{\SlopeLength}{10mm}%斜面长度 仅指水平方向 单位mm
\pgfmathsetmacro{\SlopeV}{\SlopeLength*sin(\Inclination)}%斜面竖直高度
\pgfmathsetmacro{\SlopeH}{\SlopeLength*cos(\Inclination)}%斜面水平宽度
\pgfmathsetmacro{\SlopeTriangleHeight}{\SlopeH*sin(\Inclination)}%斜面斜边高度
\pgfmathsetmacro{\CirclePointSize}{0.1}
\pgfmathsetmacro{\TheSensorSize}{0.1}

\pgfmathsetmacro{\FrameMargin}{0mm}%页边距


\pgfmathsetmacro{\CanvasW}{1.9mm}%画布左右边缘增加 \CanvasW  \m=22时 1.65
\pgfmathsetmacro{\CanvasH}{0.85mm}%画布上下边缘增加 \CanvasH
%%%%%%%%%%%%%%%%%%%%%%%%%%%%%%%%%%%%%%%%%%%%%%%%%%%%%%%%%%%%关于滑块
\pgfmathsetmacro{\TheBlockScaleRatio}{0.2}%滑块缩放因子
\pgfmathsetmacro{\TheBlockMainBodyLength}{\TheBlockScaleRatio*6mm}%滑块主体长度
\pgfmathsetmacro{\TheBlockMainBodyHeight}{\TheBlockScaleRatio*2.6mm}%滑块主体高度
\pgfmathsetmacro{\TheBlockHeadWidth}{\TheBlockScaleRatio*0.25mm}%挡块宽度
\pgfmathsetmacro{\TheBlockHeadHeight}{\TheBlockScaleRatio*1.5mm}%挡块高度
\pgfmathsetmacro{\TheBlockHeadAndEdgeDistance}{\TheBlockScaleRatio*2.2mm}%挡块边缘距滑块主体边缘距离
\pgfmathsetmacro{\TheBlockHeadsDistance}{\TheBlockMainBodyLength-2*(\TheBlockHeadWidth+\TheBlockHeadAndEdgeDistance)}%挡块间距离

\pgfmathsetmacro{\TheTopAndRealsePointDistance}{.1mm}%释放点距顶点的长度
\pgfmathsetmacro{\TheRightAndStopPointDistance}{0.3mm}%斜面右顶点距停止点的斜面距离


\pgfmathsetmacro{\GravityAcceleration}{9.8} %重力加速度
\pgfmathsetmacro{\ThePZero}{\SlopeLength/4} %第一个光电门位置(与斜面顶点距离)
\pgfmathsetmacro{\ThePZeroPSecondDistance}{\SlopeLength/4} %两光电门间距离
\pgfmathsetmacro{\TheOffsetOfPAndBlockHeadCenter}{0.0048mm} %光电门与挡块中心并不重合(挡块中心超前了),需要补偿

\def\TheAnimationFPS{10}
\def\TheAnimationFirstStepFrames{5}%第一步 帧数
\def\TheAnimationSecondStepFrames{1}%第二步 帧数   只有 一帧
\def\TheAnimationThirdStepFrames{5}%第三步 帧数
\def\TheAnimationFourthStepFrames{1}%第四步 帧数     只有 一帧
\def\TheAnimationFifthStepFrames{5}%第五步 帧数

\pgfmathsetmacro{\AnimationTotalFrame}{int(\TheAnimationFirstStepFrames+\TheAnimationSecondStepFrames+\TheAnimationThirdStepFrames+\TheAnimationFourthStepFrames+\TheAnimationFifthStepFrames)}
\begin{document}
             \begin{animateinline}[controls,%
                                  %autoplay,%
                                  %loop,%
                                  buttonsize=1.8em,%
                                  %poster=1,%
                                  method=ocg,%
                                  label=TheBlckRealseOnSlope,%
                                  ]{\TheAnimationFPS}%
            \multiframe{18}{iCount=0+1}{
            \begin{tikzpicture}%
               %\DrawTheSlopeWithPattern{x偏移量}{y偏移量}
            \DrawTheSlopeWithPattern{0}{0}
            %旋转http://tex.stackexchange.com/questions/142425/rotate-a-path-around-a-point
%http://tex.stackexchange.com/questions/19348/how-to-draw-a-line-passing-through-a-point-and-perpendicular-to-another
% \foreach \i in {2,3,...,8}
%             \coordinate (BlockRotateRoundPoint\i) at ($(SlopeTop)!(SlopeOrign)!(SlopeRight)+(\i*)$);%斜边高线与斜边交点
%             \coordinate (TheBlockReleasePoint) at
%              %%%特别注意要作旋转辅助点,还有要使用内置计算提高鲁棒性,例如BlockRotateRoundPoint的计算
%            \begin{scope}[rotate around={(-\Inclination):(BlockRotateRoundPoint)},shift={(BlockRotateRoundPoint)}]
%             \draw [fill=black] (BlockRotateRoundPoint) circle (0.5);
%                \DrawTheBlockWithPattern{0}{0}
%            \end{scope}
             \coordinate (TheSlopeHeightPoint) at ($(SlopeTop)!(SlopeOrign)!(SlopeRight)$);%斜边高线与斜边交点
             \calcLength(TheSlopeHeightPoint,SlopeTop){TopAndTheSlopeHeightPointDistance} %三角形顶点到斜边中线点的距离
             \pgfmathsetmacro{\TheBlockReleasePointShiftXOfTheSlopeHeightPoint}{\TopAndTheSlopeHeightPointDistance*cos(\Inclination)/28.3464567}%三角形顶点到斜边中线点的距离 X 分量
             \pgfmathsetmacro{\TheBlockReleasePointShiftYOfTheSlopeHeightPoint}{\TopAndTheSlopeHeightPointDistance*sin(\Inclination)/28.3464567}%三角形顶点到斜边中线点的距离 Y 分量
             \coordinate (TheBlockReleasePoint) at ($(TheSlopeHeightPoint)+(-\TheBlockReleasePointShiftXOfTheSlopeHeightPoint,\TheBlockReleasePointShiftYOfTheSlopeHeightPoint)+({\TheTopAndRealsePointDistance*cos(\Inclination)},{-\TheTopAndRealsePointDistance*sin(\Inclination)})$);              %滑块释放点 (滑块左下角所处的位置)
             \coordinate (TheBlockStopPoint) at ($(SlopeRight)+({-\TheRightAndStopPointDistance*cos(\Inclination)},{\TheRightAndStopPointDistance*sin(\Inclination)})$);%滑块停止点

             %%%%%%%%%%%%%%%%%%%%%%%光电门位置确定
             %过释放点 作  滑块过斜面时两挡块中心轨迹线的垂线  交点为 分段运动的参考点准备点 ,同时也是释放点垂直于斜面的正上方,与两挡块中心的连线重合  -----------公式   (滑块主体高度+挡块高度/2)等于与释放点的距离   \TheBlockMainBodyHeight+\TheBlockHeadHeight/2
             \coordinate (TheBlockStepMotionRefPointPrep) at ($(TheBlockReleasePoint)+({(\TheBlockMainBodyHeight+\TheBlockHeadHeight/2)*cos(90-\Inclination)},{(\TheBlockMainBodyHeight+\TheBlockHeadHeight/2)*sin(90-\Inclination)})$);
             %当滑块位于释放点时 右挡块 中心的位置   分段运动的参考点  --------------斜面方向长度=滑块主体长度-右挡块中心到滑块右面的距离    (\TheBlockMainBodyLength-(\TheBlockHeadWidth/2+\TheBlockHeadAndEdgeDistance))
             \coordinate (TheRightHeadCenterLocationWhenRealse) at ($(TheBlockStepMotionRefPointPrep)+({(\TheBlockMainBodyLength-(\TheBlockHeadWidth/2+\TheBlockHeadAndEdgeDistance))*cos(\Inclination)},{-(\TheBlockMainBodyLength-(\TheBlockHeadWidth/2+\TheBlockHeadAndEdgeDistance))*sin(\Inclination)})$);
              %两个光电门 位置
             \coordinate (ThePZeroLocation) at ($(TheBlockStepMotionRefPointPrep)+({\ThePZero*cos(\Inclination)},{-\ThePZero*sin(\Inclination)})+({-\TheTopAndRealsePointDistance*cos(\Inclination)},{\TheTopAndRealsePointDistance*sin(\Inclination)})$);%%第一个光电门位置   分段运动的参考点准备点+与斜面顶点距离-释放点距顶点的长度    距离
             \coordinate (ThePSecondLocation) at ($(ThePZeroLocation)+({\ThePZeroPSecondDistance*cos(\Inclination)},{-\ThePZeroPSecondDistance*sin(\Inclination)})$);%第二个光电门位置
             %画光电门
             \draw [DraftCirclePoint] (ThePZeroLocation) circle (\TheSensorSize);
             \draw [DraftCirclePoint] (ThePSecondLocation) circle (\TheSensorSize);

             \coordinate (TheBlockStopPointShiftOnHeadCenterMoveLine) at ($(TheBlockStopPoint)+({(\TheBlockMainBodyHeight+\TheBlockHeadHeight/2)*cos(90-\Inclination)},{(\TheBlockMainBodyHeight+\TheBlockHeadHeight/2)*sin(90-\Inclination)})$);%滑块停止点 在 挡块中心运动轨迹线上的投影
\tagged{DraftCirclePoint}{
             \fill [DraftCirclePoint] (TheBlockStopPoint) circle (\CirclePointSize);
             \fill [DraftCirclePoint] (TheBlockStopPointShiftOnHeadCenterMoveLine) circle (\CirclePointSize);
             }
\tagged{AuxiliaryLine}{
             \draw [AuxiliaryLineStyle] (TheBlockStopPointShiftOnHeadCenterMoveLine)--(TheBlockStopPoint);
             }
\tagged{DraftCirclePoint}{
        \foreach \ThePoint in {(TheSlopeHeightPoint),(TheBlockReleasePoint),(TheBlockStepMotionRefPointPrep),(TheRightHeadCenterLocationWhenRealse)}{
	    \fill \ThePoint [red] circle (\CirclePointSize);}
	}
\tagged{AuxiliaryLine}{
      \draw [AuxiliaryLineStyle] (TheSlopeHeightPoint)--(SlopeOrign); %斜面高线 辅助线
      \draw [AuxiliaryLineStyle] (TheBlockReleasePoint)--(TheBlockStepMotionRefPointPrep); %释放点辅助线
      \draw [AuxiliaryLineStyle,ExtendedLine=30cm] (TheBlockStepMotionRefPointPrep)--(TheRightHeadCenterLocationWhenRealse);%挡块中心点辅助
}
             \calcLength(TheBlockReleasePoint,TheBlockStopPoint){TheBlockTravelLength} %释放点与停止点间的距离

              %计算每步需要移动的总长
             \pgfmathsetmacro{\TheDistanceOfTwoBlockHeadCenter}{\TheBlockMainBodyLength-2*(\TheBlockHeadWidth/2+\TheBlockHeadAndEdgeDistance)} %滑块两挡块中心间距
             \calcLength(TheRightHeadCenterLocationWhenRealse,ThePZeroLocation){FirstStepMoveLength} %第一步需要移动的长度   返回值为pt /28.3464567
             \pgfmathsetmacro{\SecondStepMoveLength}{\TheDistanceOfTwoBlockHeadCenter}%第二步需要移动的长度   单位为mm
             \pgfmathsetmacro{\ThirdStepMoveLength}{\ThePZeroPSecondDistance-\TheDistanceOfTwoBlockHeadCenter}%第三步需要移动的长度   单位为mm
             \pgfmathsetmacro{\FouthStepMoveLength}{\SecondStepMoveLength}%第四步需要移动的长度  与第二步相同 单位为mm
             \calcLength(ThePSecondLocation,TheBlockStopPointShiftOnHeadCenterMoveLine){AfterThirdStepMoveLength}%第三步后移动总长度 单位为pt    %
             \pgfmathsetmacro{\FitfhStepMoveLength}{\AfterThirdStepMoveLength/28.3464567-\FouthStepMoveLength}%第四步后(第五步)需要移动的长度 单位为mm

               %计算每帧需要移动的长度
             \pgfmathsetmacro{\SecondStepMoveLengthPerFrame}{2*(\FirstStepMoveLength/28.3464567)/(\GravityAcceleration*pow(\TheAnimationFirstStepFrames,2))}   %第一步每帧移动长度 S=gt^2/2 => 平均每次移动 2S/gt^2
             %第一步
                    \pgfmathparse{int(\TheAnimationFirstStepFrames+1)}\edef\StoreResult{\pgfmathresult}%ifnumcomp 没有 <= 选项,因此先加1%http://tex.stackexchange.com/questions/337831/pgfmathparse-basic-usage
      \ifnumcomp{\iCount}{<}{\StoreResult}{%第一步移动  \TheAnimationFirstStepFrames 帧
                      \pgfmathsetmacro{\FirstStepMoveLengthPerFrame}{2*(\FirstStepMoveLength/28.3464567)/(\GravityAcceleration*pow(\TheAnimationFirstStepFrames,2))}   %第一步每帧移动长度 S=gt^2/2 => 平均每次移动 2S/gt^2
                      \pgfmathsetmacro{\CurrentTravelStepLength}{\FirstStepMoveLengthPerFrame}
                      \pgfmathsetmacro{\AfterReleaseTheBlockShiftXFormStart}{((\CurrentTravelStepLength*pow(\iCount,2)*\GravityAcceleration)/2-\TheOffsetOfPAndBlockHeadCenter)*cos(\Inclination)}%释放后与滑块左下角与释放点距离 X 分量
                      \pgfmathsetmacro{\AfterReleaseTheBlockShiftYFormStart}{((\CurrentTravelStepLength*pow(\iCount,2)*\GravityAcceleration)/2-\TheOffsetOfPAndBlockHeadCenter)*sin(\Inclination)}%%释放后与滑块左下角与释放点距离  Y 分量
                      \xdef\AfterReleaseTheBlockShiftXFormStart{\AfterReleaseTheBlockShiftXFormStart}
                      \xdef\AfterReleaseTheBlockShiftYFormStart{\AfterReleaseTheBlockShiftYFormStart}
                }{%第一步判断
             %第二步
                \pgfmathparse{int(\TheAnimationFirstStepFrames+\TheAnimationSecondStepFrames+1)}\edef\StoreResult{\pgfmathresult}%ifnumcomp 没有 <= 选项,因此先加1%http://tex.stackexchange.com/questions/337831/pgfmathparse-basic-usage
      \ifnumcomp{\iCount}{<}{\StoreResult}{%第二步移动 \TheAnimationSecondStepFrames 帧
                      \calcLength(ThePZeroLocation,TheRightHeadCenterLocationWhenRealse){tmpLength}%第一光电门与   分段运动的参考点 间的距离  pt  %
                      \pgfmathsetmacro{\SecondStepMoveLengthPerFrame}{\tmpLength/28.3464567+\TheDistanceOfTwoBlockHeadCenter-\TheOffsetOfPAndBlockHeadCenter}   %第二步(只有一帧)移动长度 S=第一步移动长度+两挡块中心距离 + 补偿
                      \pgfmathsetmacro{\CurrentTravelStepLength}{\SecondStepMoveLengthPerFrame}
                      \pgfmathsetmacro{\AfterReleaseTheBlockShiftXFormStart}{\CurrentTravelStepLength*cos(\Inclination)}%释放后与滑块左下角与释放点距离 X 分量
                      \pgfmathsetmacro{\AfterReleaseTheBlockShiftYFormStart}{\CurrentTravelStepLength*sin(\Inclination)}%%释放后与滑块左下角与释放点距离  Y 分量
                      \xdef\AfterReleaseTheBlockShiftXFormStart{\AfterReleaseTheBlockShiftXFormStart}
                      \xdef\AfterReleaseTheBlockShiftYFormStart{\AfterReleaseTheBlockShiftYFormStart}
                }{%第二步判断
             %第三步
                \pgfmathparse{int(\TheAnimationFirstStepFrames+\TheAnimationSecondStepFrames+\TheAnimationThirdStepFrames+1)}\edef\StoreResult{\pgfmathresult}%ifnumcomp 没有 <= 选项,因此先加1%http://tex.stackexchange.com/questions/337831/pgfmathparse-basic-usage
     \ifnumcomp{\iCount}{<}{\StoreResult}{%第三步移动   \TheAnimationThirdStepFrames 帧
                      \calcLength(ThePSecondLocation,TheRightHeadCenterLocationWhenRealse){tmpLength}%第二光电门与   分段运动的参考点 间的距离  pt  %
                      \pgfmathsetmacro{\ThirdStepMoveLengthPerFrame}{2*(\tmpLength/28.3464567)/(\GravityAcceleration*pow((\TheAnimationFirstStepFrames+\TheAnimationSecondStepFrames+\TheAnimationThirdStepFrames),2))}   %第三步移动总长度(相较分段运动的参考点)长度 S= 第一光电门与   分段运动的参考点 间的距离            第三步每帧移动长度 S=gt^2/2 => 平均每次移动 2S/gt^2
                      \pgfmathsetmacro{\CurrentTravelStepLength}{\ThirdStepMoveLengthPerFrame}
                     \calcLength(ThePZeroLocation,TheRightHeadCenterLocationWhenRealse){tmpLengthFst}%第一步移动长度  pt  %
                      \pgfmathsetmacro{\SecondStepMoveLengthPerFrame}{\tmpLengthFst/28.3464567+\TheDistanceOfTwoBlockHeadCenter-\TheOffsetOfPAndBlockHeadCenter}   %第二步移动长度 S=第一步移动长度+两挡块中心距离 + 补偿
                      %%第三步内部判断   要去掉长度单位再比较
                    \pgfmathsetmacro{\CurrentTravelLengthx}{((\CurrentTravelStepLength*pow(\iCount,2)*\GravityAcceleration)/2-\TheOffsetOfPAndBlockHeadCenter)}% 当前移动长度
                    \pgfmathsetmacro{\CurrentTravelLengthxWithoutUnits}{scalar(\CurrentTravelLengthx)}% 当前移动长度无单位
                    \pgfmathsetmacro{\SecondStepMoveLengthPerFrameWithoutUnits}{scalar(\SecondStepMoveLengthPerFrame)} %第二步移动长度 无单位
                    \ifthenelse{\lengthtest{\CurrentTravelLengthxWithoutUnits pt  < \SecondStepMoveLengthPerFrameWithoutUnits pt}}{ %如果此时计算出的移动长度反倒小于上一步的移动长度,那么把移动时间(\iCount) 加1
                         \pgfmathsetmacro{\iCountx}{int(\iCount+1)}
                         \xdef\iCount{\iCountx} }{} %后面这对大括号绝对不能少,如果缺少,会产生非预期的图像
                      \pgfmathsetmacro{\AfterReleaseTheBlockShiftXFormStart}{((\CurrentTravelStepLength*pow(\iCount,2)*\GravityAcceleration)/2-3*\TheOffsetOfPAndBlockHeadCenter)*cos(\Inclination)}%释放后与滑块左下角与释放点距离 X 分量
                      \pgfmathsetmacro{\AfterReleaseTheBlockShiftYFormStart}{((\CurrentTravelStepLength*pow(\iCount,2)*\GravityAcceleration)/2-3*\TheOffsetOfPAndBlockHeadCenter)*sin(\Inclination)}%%释放后与滑块左下角与释放点距离  Y 分量
                      \xdef\AfterReleaseTheBlockShiftXFormStart{\AfterReleaseTheBlockShiftXFormStart}
                      \xdef\AfterReleaseTheBlockShiftYFormStart{\AfterReleaseTheBlockShiftYFormStart}%%%%%%%%
                }{%第三步判断
              %第四步
                 \pgfmathparse{int(\TheAnimationFirstStepFrames+\TheAnimationSecondStepFrames+\TheAnimationThirdStepFrames+\TheAnimationFourthStepFrames+1)}\edef\StoreResult{\pgfmathresult}%ifnumcomp 没有 <= 选项,因此先加1%http://tex.stackexchange.com/questions/337831/pgfmathparse-basic-usage
                \ifnumcomp{\iCount}{<}{\StoreResult}{%第三步移动   \TheAnimationFourthStepFrames 帧
                      \calcLength(ThePSecondLocation,TheRightHeadCenterLocationWhenRealse){tmpLength}%第二光电门与   分段运动的参考点 间的距离  pt  %
                      \pgfmathsetmacro{\FourthStepMoveLengthPerFrame}{\tmpLength/28.3464567+\TheDistanceOfTwoBlockHeadCenter-3*\TheOffsetOfPAndBlockHeadCenter}   %第四步(只有一帧)移动长度 S=第三步移动长度+两挡块中心距离 + 补偿
                      \pgfmathsetmacro{\CurrentTravelStepLength}{\FourthStepMoveLengthPerFrame}
                      \pgfmathsetmacro{\AfterReleaseTheBlockShiftXFormStart}{\CurrentTravelStepLength*cos(\Inclination)}%释放后与滑块左下角与释放点距离 X 分量
                      \pgfmathsetmacro{\AfterReleaseTheBlockShiftYFormStart}{\CurrentTravelStepLength*sin(\Inclination)}%%释放后与滑块左下角与释放点距离  Y 分量
                      \xdef\AfterReleaseTheBlockShiftXFormStart{\AfterReleaseTheBlockShiftXFormStart}
                      \xdef\AfterReleaseTheBlockShiftYFormStart{\AfterReleaseTheBlockShiftYFormStart}
                }{%第四步判断
                %第五步
                \pgfmathparse{int(\TheAnimationFirstStepFrames+\TheAnimationSecondStepFrames+\TheAnimationThirdStepFrames+\TheAnimationFourthStepFrames+\TheAnimationFifthStepFrames+1)}\edef\StoreResult{\pgfmathresult}%ifnumcomp 没有 <= 选项,因此先加1%http://tex.stackexchange.com/questions/337831/pgfmathparse-basic-usage
     \ifnumcomp{\iCount}{<}{\StoreResult}{%第五步移动   \TheAnimationFifthStepFrames 帧
                      \calcLength(TheRightHeadCenterLocationWhenRealse,TheBlockStopPointShiftOnHeadCenterMoveLine){tmpLength}%停止点与   分段运动的参考点 间的距离  pt  %
                      \pgfmathsetmacro{\FifthStepMoveLengthPerFrame}{2*(\tmpLength/28.3464567-(\TheBlockHeadWidth/2+\TheBlockHeadAndEdgeDistance))/(\GravityAcceleration*pow((\TheAnimationFirstStepFrames+\TheAnimationSecondStepFrames+\TheAnimationThirdStepFrames+\TheAnimationFourthStepFrames+\TheAnimationFifthStepFrames),2))}   %第五步移动总长度(相较分段运动的参考点)长度 S= 停止点与 分段运动的参考点 间的距离 -右挡块中心与滑块右边缘的距离           第五步每帧移动长度 S=gt^2/2 => 平均每次移动 2S/gt^2
                      \pgfmathsetmacro{\CurrentTravelStepLength}{\FifthStepMoveLengthPerFrame}
                     \calcLength(ThePSecondLocation,TheRightHeadCenterLocationWhenRealse){tmpLengthThi}%第三步移动长度  pt  %
                      \pgfmathsetmacro{\FourthStepMoveLengthPerFrame}{\tmpLengthThi/28.3464567+\TheDistanceOfTwoBlockHeadCenter-3*\TheOffsetOfPAndBlockHeadCenter}   %第四步移动长度 S=第三步移动长度+两挡块中心距离 + 补偿
                      %%第三步内部判断   要去掉长度单位再比较
                    \pgfmathsetmacro{\CurrentTravelLengthx}{((\CurrentTravelStepLength*pow(\iCount,2)*\GravityAcceleration)/2-3*\TheOffsetOfPAndBlockHeadCenter)}% 当前移动长度
                    \pgfmathsetmacro{\CurrentTravelLengthxWithoutUnits}{scalar(\CurrentTravelLengthx)}% 当前移动长度无单位
                    \pgfmathsetmacro{\FourthStepMoveLengthPerFrameWithoutUnits}{scalar(\FourthStepMoveLengthPerFrame)} %第四步移动长度 无单位
                    \ifthenelse{\lengthtest{\CurrentTravelLengthxWithoutUnits pt  < \FourthStepMoveLengthPerFrameWithoutUnits pt}}{ %如果此时计算出的移动长度反倒小于上一步的移动长度,那么把移动时间(\iCount) 加1
                         \pgfmathsetmacro{\iCountx}{int(\iCount+1)}
                         \xdef\iCount{\iCountx} }{} %后面这对大括号绝对不能少,如果缺少,会产生非预期的图像
                      \pgfmathsetmacro{\AfterReleaseTheBlockShiftXFormStart}{((\CurrentTravelStepLength*pow(\iCount,2)*\GravityAcceleration)/2-5*\TheOffsetOfPAndBlockHeadCenter)*cos(\Inclination)}%释放后与滑块左下角与释放点距离 X 分量
                      \pgfmathsetmacro{\AfterReleaseTheBlockShiftYFormStart}{((\CurrentTravelStepLength*pow(\iCount,2)*\GravityAcceleration)/2-5*\TheOffsetOfPAndBlockHeadCenter)*sin(\Inclination)}%%释放后与滑块左下角与释放点距离  Y 分量
                      \xdef\AfterReleaseTheBlockShiftXFormStart{\AfterReleaseTheBlockShiftXFormStart}
                      \xdef\AfterReleaseTheBlockShiftYFormStart{\AfterReleaseTheBlockShiftYFormStart}%%%%%%%%
                }{%第五步判断
                }
                }%第四步判断
                }%第三步判断
                }%第二步判断
                }%第一步判断


             %
              \coordinate (TheBlockCurrentLocationPoint\iCount) at ($(TheBlockReleasePoint)+(\AfterReleaseTheBlockShiftXFormStart,-\AfterReleaseTheBlockShiftYFormStart)$);
               %%%特别注意要作旋转辅助点,还有要使用内置计算提高鲁棒性,例如BlockRotateRoundPoint的计算

             \begin{scope}[rotate around={(-\Inclination):(TheBlockCurrentLocationPoint\iCount)},shift={(TheBlockCurrentLocationPoint\iCount)}]
             %\draw [fill=black] (TheBlockCurrentLocationPoint\iCount) circle (0.5mm);
                %\DrawTheBlockWithPattern{0}{0}
                \DrawTheBlockWithPatternAndNumbered{0}{0}{\iCount}
             \end{scope}



            \end{tikzpicture}
            }%
            \end{animateinline}%
\end{document}

%%%%%%%%%%%%%%%%%%%%%%%%%%%%%%%%%%%%%%%%%%%%%%%%%%%%%%%%%%%%%%%%%%%%%%%%%%%%%%%%
%%%%%%%%%%%%%%%%%%%%%%%%%%%%%%%%%%%%%%%%%%%%%%%%%%%%%%%%%%%%%%%%%%%%%%%%%%%%%%%%
%%%%%%%%%%%%%%%%%%%%%%%%%%%%%%%%%%%%%%%%%%%%%%%%%%%%%%%%%%%%%%%%%%%%%%%%%%%%%%%%
%%%%%%%%%%%%%%%%%%%%%%%%%%%%%%%%%%%%%%%%%%%%%%%%%%%%%%%%%%%%%%%%%%%%%%%%%%%%%%%%
%\usetag{0}%%%%%只有玻璃和透镜 光路
%%\usetag{1}%%%%装置 和牛顿环的像
%%\usetag{2}%%%%%%%仅装置但包含辅助线
%%\usetag{3}%%%%%怎么数环            %pdf动画单独一页standalone 选项为不要tikz
%    %\usetag{3ToSWF}%pdf动画每帧为一页单独输出  此时 standalone 选项必须加 tikz ,要转换swf文件时才使用此行
%%\usetag{4}%%%%dm ds lm ls 关系
%%\usetag{5}%怎么读数              %pdf动画单独一页standalone 选项为不要tikz
%    %\usetag{5ToSWF}%pdf动画每帧为一页单独输出    此时 standalone 选项必须加 tikz ,要转换swf文件时才使用此行
%%%%%%%%%%%%%%%%%%%%%%%%%%%%%%%%%%%%%%%%%%%%%%%%%%%%%%%%%%%%%%%%%%%%%%%%%%%%%%%%
%%%%%%%%%%%%%%%%%%%%%%%%%%%%%%%%%%%%%%%%%%%%%%%%%%%%%%%%%%%%%%%%%%%%%%%%%%%%%%%%
%%%%%%%%%%%%%%%%%%%%%%%%%%%%%%%%%%%%%%%%%%%%%%%%%%%%%%%%%%%%%%%%%%%%%%%%%%%%%%%%
%%%%%%%%%%%%%%%%%%%%%%%%%%%%%%%%%%%%%%%%%%%%%%%%%%%%%%%%%%%%%%%%%%%%%%%%%%%%%%%%
%
%
%\newcommand{\Logo}{\scalebox{0.05}{\includegraphics{DEConfig/CUGLogo}}}%校徽
%
%
%\begin{document}%
%%
%\pagecolor{yellow!50}%页面颜色 50%黄色
%%%%%%%%%%%%%%%%%%%%%%%%%%%%%%%%%%%%%%%%%%%%%%%%%%%%%%%%%%%%%%%%%%%%%%%%%%%%%%%%%%%%%%%%%%%%%%%%%%%%%%%%%%%%%%%%%%%%%%%%%%%%%
%%%%%%%%%%%%%%%%%%%%%%%%%%%%%%%%%%%%%%%%%%%%%%%%%%%%%%%%%%%%%%%%%%%%%%%%%%%%%%%%%%%%%%%%%%%%%%%%%%%%%%%%%%%%%%%%%%%%%%%%%%%%%
%%%%%%%%%%%%%%%%%%%%%%%%%%%%%%%%%%%%%%%%%%%%%%%%%%%%%%%%%%%%%%%%%%%%%%%%%%%%%%%%%%%%%%%%%%%%%%%%%%%%%%%%%%%%%%%%%%%%%%%%%%%%%
%\tagged{0}{%只有玻璃和透镜
%            \begin{tikzpicture}%
%            \path[use as bounding box] (-6.2,-5.8) rectangle (6.2,4.85); %只显示该范围内图形
%            %\draw (-8,-8) to[grid with coordinates] (7,6); %坐标 网格辅助线
%            %%\DrawAFullNewtonRingsNormal
%            \DrawRingOnlyCoordinate%
%            %%\DrawLenGlassWithRings
%            \node at ($(0,-1.75*\CanvasH)+(CoorL\MaxRings|-CoorUU\MaxRings)$)[opacity=0.62,anchor=center] (Logo){\Logo};%
%            \node at (Logo)[yshift=20,scale=0.5]{\TheCUG};%
%            \DrawLenGlassWithoutAuxiliaryLine%
%            \end{tikzpicture}%
%}%
%%%%%%%%%%%%%%%%%%%%%%%%%%%%%%%%%%%%%%%%%%%%%%%%%%%%%%%%%%%%%%%%%%%%%%%%%%%%%%%%%%%%%%%%%%%%%%%%%%%%%%%%%%%%%%%%%%%%%%%%%%%%%
%%%%%%%%%%%%%%%%%%%%%%%%%%%%%%%%%%%%%%%%%%%%%%%%%%%%%%%%%%%%%%%%%%%%%%%%%%%%%%%%%%%%%%%%%%%%%%%%%%%%%%%%%%%%%%%%%%%%%%%%%%%%%
%%%%%%%%%%%%%%%%%%%%%%%%%%%%%%%%%%%%%%%%%%%%%%%%%%%%%%%%%%%%%%%%%%%%%%%%%%%%%%%%%%%%%%%%%%%%%%%%%%%%%%%%%%%%%%%%%%%%%%%%%%%%%
%%%%%%%%%%%%%%%%%%%%%%%%%%%%%%%%%%%%%%%%%%%%%%%%%%%%%%%%%%%%%%%%%%%%%%%%%%%%%%%%%%%%%%%%%%%%%%%%%%%%%%%%%%%%%%%%%%%%%%%%%%%%%
%\tagged{1}{%
%            %%%%%%%%%%%%%%%%%%%%%%%%%%%%牛顿环和装置
%            \begin{tikzpicture}%
%            \path[use as bounding box] (-6.0,-5.45) rectangle (6.0,21.45); %只显示该范围内图形
%            %\draw (-7,-5.6) to[grid with coordinates] (7,22); %坐标 网格辅助线
%            \DrawAFullNewtonRingsNormal%
%            \node at ($(0.15*\CanvasW,17.76*\CanvasH)+(CoorL\MaxRings|-CoorUU\MaxRings)$)[opacity=0.62,anchor=center](Logo) {\Logo};%
%            \node at (Logo)[yshift=20,scale=0.5]{\TheCUG};%
%            \DrawLenGlassWithRings%
%            \PutAnEye%
%            \end{tikzpicture}%
%}%
%%%%%%%%%%%%%%%%%%%%%%%%%%%%%%%%%%%%%%%%%%%%%%%%%%%%%%%%%%%%%%%%%%%%%%%%%%%%%%%%%%%%%%%%%%%%%%%%%%%%%%%%%%%%%%%%%%%%%%%%%%%%%
%%%%%%%%%%%%%%%%%%%%%%%%%%%%%%%%%%%%%%%%%%%%%%%%%%%%%%%%%%%%%%%%%%%%%%%%%%%%%%%%%%%%%%%%%%%%%%%%%%%%%%%%%%%%%%%%%%%%%%%%%%%%%
%%%%%%%%%%%%%%%%%%%%%%%%%%%%%%%%%%%%%%%%%%%%%%%%%%%%%%%%%%%%%%%%%%%%%%%%%%%%%%%%%%%%%%%%%%%%%%%%%%%%%%%%%%%%%%%%%%%%%%%%%%%%%
%\tagged{2}{%
%            %%%%%%%%%%%%%%%%%%%%%%%%%%%%%%装置和辅助线
%            \begin{tikzpicture}%
%            \path[use as bounding box] (-6.0,-5.62) rectangle (6.0,10.29); %只显示该范围内图形
%            %\draw (-7,-5.6) to[grid with coordinates] (7,10.25); %坐标 网格辅助线
%            \DrawRingOnlyCoordinate%
%            \node at ($(0.46*\CanvasW,-0.95*\CanvasH)+(CoorL\MaxRings|-CoorUU\MaxRings)$)[xshift=-18,yshift=135,opacity=0.62,anchor=center] (Logo){\Logo};%
%            \node at (Logo)[yshift=20,scale=0.5]{\TheCUG};%
%            \DrawLenGlassOnlyWithAuxiliaryLine%
%            \end{tikzpicture}%
%}%
%%
%%
%%%%%%%%%%%%%%%%%%%%%%%%%%%%%%%%%%%%%%%%%%%%%%%%%%%%%%%%%%%%%%%%%%%%%%%%%%%%%%%%%%%%%%%%%%%%%%%%%%%%%%%%%%%%%%%%%%%%%%%%%%%%%
%%%%%%%%%%%%%%%%%%%%%%%%%%%%%%%%%%%%%%%%%%%%%%%%%%%%%%%%%%%%%%%%%%%%%%%%%%%%%%%%%%%%%%%%%%%%%%%%%%%%%%%%%%%%%%%%%%%%%%%%%%%%%
%%%%%%%%%%%%%%%%%%%%%%%%%%%%%%%%%%%%%%%%%%%%%%%%%%%%%%%%%%%%%%%%%%%%%%%%%%%%%%%%%%%%%%%%%%%%%%%%%%%%%%%%%%%%%%%%%%%%%%%%%%%%%
%\pgfmathsetmacro{\IndicateNumberTotalFrame}{int(\PointOutRingMax+2)}%演示怎么数环时总帧数
%\tagged{3}{%
%%%%%%%%%%%%%%%%%%%%%%%%%%%%怎么数环
%                                                %pdf动画
%            \begin{animateinline}[controls,%
%                                  %autoplay,%
%                                  %loop,%
%                                  buttonsize=1.8em,%
%                                  %poster=1,%
%                                  method=ocg,%
%                                  label=RingsRead,%
%                                  ]{0.75}%
%            \multiframe{\IndicateNumberTotalFrame}{n=1+1}{%\IndicateNumberTotalFrame
%                                                            \begin{tikzpicture}[]%
%                                                            \path[use as bounding box] (-5.45,-5.8) rectangle (5.45,5.45); %只显示该范围内图形
%                                                            %\draw (-6.2,-5.8) to[grid with coordinates] (6.2,5.8); %坐标 网格辅助线
%                                                            \DrawAFullNewtonRingsNormal%
%                                                            \node at ($(0.42*\CanvasW,-1.06*\CanvasH)+(CoorL\MaxRings|-CoorUU\MaxRings)$)[opacity=0.62,anchor=center] (Logo){\Logo};%
%                                                            \node at (Logo)[yshift=20,scale=0.5]{\TheCUG};
%                                                            \IndicateRingNumber%
%                                                            \end{tikzpicture}%
%                                                        }%
%            \end{animateinline}%
%}%
%%%%%%%%%%%%%%%%%%%%%%%%%%%%%%%%%%%%%%%%%%%%%%%%%%%%%%%%%%%%%%%%%%%%%%%%%%%%%%%%%%%%%%%%%%%%%%%%%%%%%%%%%%%%%%%%%%%%%%%%%%%%%
%\tagged{3ToSWF}{%
%%%%%%%%%%%%%%%%%%%%%%%%%%%%怎么数环
%                                                %swf动画
%            \foreach \i in {1,...,\IndicateNumberTotalFrame}{%\IndicateNumberTotalFrame
%                                                            \begin{tikzpicture}[]%
%                                                            \path[use as bounding box] (-5.45,-6.72) rectangle (5.45,5.5); %只显示该范围内图形
%                                                            %\draw (-6.2,-5.8) to[grid with coordinates] (6.2,5.8); %坐标 网格辅助线
%                                                            \DrawAFullNewtonRingsNormal%
%                                                            \node at ($(0.42*\CanvasW,-\CanvasH)+(CoorL\MaxRings|-CoorUU\MaxRings)$)[opacity=0.62,anchor=center] (Logo){\Logo};%
%                                                            \node at (Logo)[yshift=20,scale=0.5]{\TheCUG};%
%                                                            \IndicateRingNumber%
%                                                            \end{tikzpicture}%
%                                                            }%
%}%
%%
%%%%%%%%%%%%%%%%%%%%%%%%%%%%%%%%%%%%%%%%%%%%%%%%%%%%%%%%%%%%%%%%%%%%%%%%%%%%%%%%%%%%%%%%%%%%%%%%%%%%%%%%%%%%%%%%%%%%%%%%%%%%%
%%%%%%%%%%%%%%%%%%%%%%%%%%%%%%%%%%%%%%%%%%%%%%%%%%%%%%%%%%%%%%%%%%%%%%%%%%%%%%%%%%%%%%%%%%%%%%%%%%%%%%%%%%%%%%%%%%%%%%%%%%%%%
%%%%%%%%%%%%%%%%%%%%%%%%%%%%%%%%%%%%%%%%%%%%%%%%%%%%%%%%%%%%%%%%%%%%%%%%%%%%%%%%%%%%%%%%%%%%%%%%%%%%%%%%%%%%%%%%%%%%%%%%%%%%%
%\tagged{4}{%
%%%%%%%%%%%%%%%%%%%%%%%%%%%dm ds lm ls 关系
%            \begin{tikzpicture}%
%            \path[use as bounding box] (-6.55,-6.4) rectangle (5.8,6.48); %只显示该范围内图形
%            %\draw (-7.0,-6.76) to[grid with coordinates] (5.4,6.48); %坐标 网格辅助线
%            \DrawRingMAndS%
%            \node at ($(-0.15*\CanvasW,0)+(CoorL\MaxRings|-CoorUU\MaxRings)$) [yshift=4,opacity=0.62,anchor=center] (Logo) {\Logo};%
%            \node at (Logo)[yshift=20,scale=0.5]{\TheCUG};%
%            \DmDsCote%
%            \LmLsCote%
%            \DmDsLmLsRelationship%
%            \end{tikzpicture}%
%}%
%%
%%%%%%%%%%%%%%%%%%%%%%%%%%%%%%%%%%%%%%%%%%%%%%%%%%%%%%%%%%%%%%%%%%%%%%%%%%%%%%%%%%%%%%%%%%%%%%%%%%%%%%%%%%%%%%%%%%%%%%%%%%%%%
%%%%%%%%%%%%%%%%%%%%%%%%%%%%%%%%%%%%%%%%%%%%%%%%%%%%%%%%%%%%%%%%%%%%%%%%%%%%%%%%%%%%%%%%%%%%%%%%%%%%%%%%%%%%%%%%%%%%%%%%%%%%%
%%%%%%%%%%%%%%%%%%%%%%%%%%%%%%%%%%%%%%%%%%%%%%%%%%%%%%%%%%%%%%%%%%%%%%%%%%%%%%%%%%%%%%%%%%%%%%%%%%%%%%%%%%%%%%%%%%%%%%%%%%%%%
%\pgfmathsetmacro{\TakeReadingTotalFrame}{int(3*(\m+2)+1)}%演示怎么读数时总帧数
%\tagged{5}{%
%%%%%%%%%%%%%%%%%%%%%%%%%%%%怎么测量dm ds
%                                               %pdf动画
%            \begin{animateinline}[controls,%
%                                  %autoplay,
%                                  %loop,
%                                  buttonsize=1.8em,%
%                                  %poster=1,%
%                                  method=ocg,%
%                                  label=TakeReading,%
%                                  ]{1}%
%            \multiframe{\TakeReadingTotalFrame}{N=1+1}{%73 %3 poster=1   \multiframe{\TakeReadingTotalFrame}{N=1+1}
%                                                \begin{tikzpicture}[spy using outlines={circle, magnification=6, connect spies,},]%%dash pattern= on 1 off 1
%                                                %\path[use as bounding box] (-6,-5.4) rectangle (6,6.48); %只显示该范围内图形
%                                                %\draw (-8,-8) to[grid with coordinates] (8,8); %坐标 网格辅助线
%                                                %\DrawAFullNewtonRingsShade%
%                                                \DrawAFullNewtonRingsNormal
%                                                %\DrawRingMAndS
%                                                \node at ($(-0.63*\CanvasW,0)+(CoorL\MaxRings|-CoorUU\MaxRings)$) [opacity=0.62,anchor=center] (Logo) {\Logo};
%                                                \node at (Logo)[yshift=20,scale=0.5]{\TheCUG};
%                                                \ShowHowToTakeReading
%                                                \end{tikzpicture}%
%                                                     }%fram结束
%            \end{animateinline}%
%}%
%%%%%%%%%%%%%%%%%%%%%%%%%%%%%%%%%%%%%%%%%%%%%%%%%%%%%%%%%%%%%%%%%%%%%%%%%%%%%%%%%%%%%%%%%%%%%%%%%%%%%%%%%%%%%%%%%%%%%%%%%%%%%
%\tagged{5ToSWF}{%
%%%%%%%%%%%%%%%%%%%%%%%%%%%%怎么测量dm ds
%                                        %swf动画
%            \foreach \i in {1,...,\TakeReadingTotalFrame}{% 1-\TakeReadingTotalFrame
%                                                            \begin{tikzpicture}[spy using outlines={circle, magnification=6, connect spies},]%dash pattern= on 1 off 1%
%                                                            %\DrawAFullNewtonRingsShade%
%                                                            \DrawAFullNewtonRingsNormal
%                                                            \node at ($(1.22*\CanvasW,-0.96*\CanvasH)-(CoorLU\MaxRings|-CoorUU\MaxRings)$) {\tagged{Draft}{\tikz\draw[fill=red,color=red] circle(2pt);}{}};%扩大画布放置播放按钮
%                                                            \node at ($(-0.63*\CanvasW,0)+(CoorL\MaxRings|-CoorUU\MaxRings)$) [opacity=0.62,anchor=center] (Logo) {\Logo};%放置Logo
%                                                            \node at (Logo)[yshift=20,scale=0.5]{\TheCUG};%放置CUG字母
%                                                            \ShowHowToTakeReading%
%                                                            \end{tikzpicture}%
%                                                            }%
%}%
%%
%\end{document} 